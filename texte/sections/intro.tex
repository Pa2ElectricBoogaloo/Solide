One of the earliest breakthrews allowed by quantum mechanics is the descritpion of metals and insulators with band theory \cite{hoddeson_development_1987} which saw light in 1930. This theory brought a microscopic understanding of the distinction between these two classes of materials and led to incredible technological advances such as the discovery of the transistor by John Bardeen and Walter Brattain in 1947 \cite{brinkman_history_1997}. Beyond its incredible success,  band theory rapidly came in contact with a myriad of intriguing quantum mechanical effects such as the integer quantum Hall effect discovered in 1980 \cite{klitzing_new_1980}. In 1982, Thouless et al. \cite{cayssol_topological_2021} figured out the topological nature of the effect and, in terms, brought topology closer to band theory. Although the integer quantum Hall effect (QH) requires a strong external magnetic field, it was theorized in 2005 by Charles Kane and Eugene Mele \cite{kane_quantum_2005} that similar topological properties could be intrinsically realized  through the quantum spin Hall effect (QSH) \cite{qi_quantum_2010}. Experiments then showed, in 2007 \cite{koenig_quantum_2007},that HgTe/CdTe
quantum wells (mercury telluride heterostructure) could produce a QSH effect. The theory and experiment of QSH effect led to a deeper classification of solids with topological band theory \cite{soluyanov_topological_nodate}. When applied to insulators, the upgraded band theory creates a separation between the trivial and the \textit{topological insulators} (TI). The latter is generally characterized by a metallic boundary and an insulating bulk \cite{moore_birth_2010} as opposed to trivial insulators which are insulating everywhere. The present review will focus on basic properties of time reversal symmetric topological insulators based on the mercury telluride example. Sec.\ref{sec:kramers} presents an overview of important ideas from topological band theory. In sec.\ref{sec:QSH}, the main properties of the QSH state are given and compared to the QH effect. Finally, a model of HgTe/CdTe
quantum is studied in sec. \ref{sec:ti_exemples}. 