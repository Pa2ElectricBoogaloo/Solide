This sections aims to describe the notion of topological invariant and its consequence on the band structure. 
\subsection{Topological equivalence of insulators}
In a periodic lattice potential, electrons are described by bloch states $\ket{n, \mathbf{k}}$ where $n$ is a dicrete quantum number and $\mathbf{k}$ is the crystal momentum in the brillouin zone \cite{shankar_topological_2018}. Each of those states is associated to an energy $E_{n, \mathbf{k}}$.  As it varies with $\mathbf{k}$, the energy sweeps a continous range called a band labeled with the number $n$. Bands are often separated by energy gaps where there are no associated states. Trivial insulators have a gapped ground state meaning that low energy excitations are forbidden by the presence of the gap. On the contrary, topological insulators have metallic (gapless edge states)\cite{kane_topological_2013} and different from trivial insulators in a fundamental way.\\ 

Slow modifications of the hamiltonian of a trivial insulator will change its band structure while leaving it in its ground state. The \textit{deformation} is said to yield equivalent insulators if the gap doesn't close \cite{kane_topological_2013}. This equvalence is topological in the same way the continuous disformation from a torus to a coffee mug is. Just like the coffee mug cannot be continuously disformed into a sphere, a trivial insulator cannot be continuously disformet into a topological insulator.

\subsection{Time reversal symmetry}



\subsection{$\mathbb{Z}_2$ invariant}