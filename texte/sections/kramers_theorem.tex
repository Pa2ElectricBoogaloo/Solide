This sections aims to describe the notion of topological invariant and its consequence on the band structure. 
\subsection{Topological equivalence of insulators}
In a periodic lattice potential, electrons are described by bloch states $\ket{n, \mathbf{k}}$ where $n$ is a dicrete quantum number and $\mathbf{k}$ is the crystal momentum in the brillouin zone \cite{shankar_topological_2018}. Each of those states is associated to an energy $E_{n, \mathbf{k}}$.  As it varies with $\mathbf{k}$, the energy sweeps a continous range called a band labeled with the number $n$. Bands are often separated by energy gaps where there are no associated states. Trivial insulators have a gapped ground state meaning that low energy excitations are forbidden by the presence of the gap. On the contrary, topological insulators have metallic (gapless edge states)\cite{kane_topological_2013} and different from trivial insulators in a fundamental way.\\ 

Slow modifications of the hamiltonian of a trivial insulator will change its band structure while leaving it in its ground state. The \textit{deformation} is said to yield equivalent insulators if the gap doesn't close \cite{kane_topological_2013}. This equvalence is topological in the same way the continuous disformation from a torus to a coffee mug is. Just like the coffee mug cannot be continuously disformed into a sphere, a trivial insulator cannot be continuously disformed into a topological insulator.\\

For the coffee mug and the sphere, a single number tells if there is topological equivalence or not. This number is called the genus (the number of holes) and it as analogues for topological insulators \cite{batra_physics_2020}. The central property of genus is that it only changes when the torus is broken into a sphere in a necessarly discontinuous way. Therefore, the genus is a topological invariant for the sphere and torus. For the QH effect the topological invariant indicating the persistance of the edge states uppon deforming the Hamiltonian is the Chern number. For the QSH effect, the $\mathbb{Z}_2$ invariant is involved \cite{kane_topological_2013}. In the following section, the link between time reversal symmetry and this last invariant is detailed.  Mixing symmetry, topological invariants and dimensionnality allows for a classification of topological insulators \cite{hasan_topological_2010}. 

\subsection{Time reversal symmetry}

Time reversal symmetry (TRS) plays an important role in QSH topological insulators: it protects the gapless edge states \cite{kane_topological_2013}. While the QH effect requiers a magnetic field to occur (breaking of TRS), the QSH effect doesn't and it preserves TRS\cite{cayssol_topological_2021}. 



\subsection{$\mathbb{Z}_2$ invariant}