% ****** Start of file apssamp.tex ******
%
%   This file is part of the APS files in the REVTeX 4.2 distribution.
%   Version 4.2a of REVTeX, December 2014
%
%   Copyright (c) 2014 The American Physical Society.
%
%   See the REVTeX 4 README file for restrictions and more information.
%
% TeX'ing this file requires that you have AMS-LaTeX 2.0 installed
% as well as the rest of the prerequisites for REVTeX 4.2
%
% See the REVTeX 4 README file
% It also requires running BibTeX. The commands are as follows:
%
%  1)  latex apssamp.tex
%  2)  bibtex apssamp
%  3)  latex apssamp.tex
%  4)  latex apssamp.tex
%

\documentclass[%
reprint,
%superscriptaddress,
%groupedaddress,
%unsortedaddress,
%runinaddress,
%frontmatterverbose, 
% preprint,
%preprintnumbers,
%nofootinbib,
%nobibnotes,
%bibnotes,
 amsmath,amssymb,
 aps,
%pra,
%prb,
%rmp,
%prstab,
%prstper,
%floatfix,
]{revtex4-2}

\usepackage{lipsum}
\usepackage{graphics}% Include figure files
\usepackage{dcolumn}% Align table columns on decimal point
\usepackage{bm}% bold math
\usepackage{hyperref}% add hypertext capabilities
%\usepackage[mathlines]{lineno}% Enable numbering of text and display math
%\linenumbers\relax % Commence numbering lines
\usepackage[inline]{asymptote}
\usepackage{svg}
\usepackage{chemformula}


%\usepackage[showframe,%Uncomment any one of the following lines to test 
%scale=0.7, marginratio={1:1, 2:3}, ignoreall,% default settings
%text={7in,10in},centering,
%margin=1.5in,
%total={6.5in,8.75in}, top=1.2in, left=0.9in, includefoot,
%height=10in,a5paper,hmargin={3cm,0.8in},
%]{geometry}

\begin{document}

\preprint{APS/123-QED}

\title{Topological Superconductivity} 

\author{Pierre-Antoine Graham}
\author{Jean-Baptiste Bertrand}

\altaffiliation{Physics Department, Sherbrooke University.}

\date{\today}

\begin{abstract}
The field of topological insulators is a recent and very broad. This review aims to be an introduction to the topic. It covers the basic topological concept required to understand the theory such as the idea of topological equivalence, Kramer's theorem and the berry phase. Some of the earliest and simplest application of the theory  are quickly explained. These cover the Integer Quantum and Quantum Spin Hall effect, and the example of the HGTE/CDTE heterostructure.
\end{abstract}

\maketitle

\tableofcontents

\listoffigures

{\footnotesize
\section{\label{sec:intro} Introduction}
The present review will focus on the basic properties of topological insulators with time-reversal symmetry. 
One of the earliest breakthrews allowed by quantum mechanics is the descritpion of metals and insulators with band theory \cite{hoddeson_development_1987} which saw light in 1930. This theory brought a microscopic understanding of the distinction between these two classes of materials and led to incredible technological advances such as the discovery of the transistor by John Bardeen and Walter Brattain in 1947 \cite{brinkman_history_1997}. Beyond its incredible success,  band theory rapidly came in contact with a myriad of intriguing quantum mechanical effects such as the integer quantum Hall effect discovered in 1980 \cite{klitzing_new_1980}. In 1982, Thouless et al. \cite{cayssol_topological_2021} figured out the topological nature of the effect and, in terms, brought topology closer to band theory. Although the integer quantum Hall effect (QH) requires a strong external magnetic field, it was theorized in 2005 by Charles Kane and Eugene Mele \cite{kane_quantum_2005} that similar topological properties could be intrinsically realized  through the quantum spin Hall effect (QSH) \cite{qi_quantum_2010}. Experiments then showed, in 2007 \cite{koenig_quantum_2007},that HgTe/CdTe
quantum wells (mercury telluride heterostructure) could produce a QSH effect. The theory and experiment of QSH effect led to a deeper classification of solids with topological band theory \cite{soluyanov_topological_nodate}. When applied to insulators, the upgraded band theory creates a separation between the trivial and the \textit{topological insulators} (TI). The latter is generally characterized by a metallic boundary and an insulating bulk \cite{moore_birth_2010} as opposed to trivial insulators which are insulating everywhere. The present review will focus on basic properties of time reversal symmetric topological insulators based on the mercury telluride example. Sec.\ref{sec:kramers} presents an overview of important ideas from topological band theory. In sec.\ref{sec:QSH}, the main properties of the QSH state are given and compared to the QH effect. Finally, a model of HgTe/CdTe
quantum is studied in sec. \ref{sec:ti_exemples}. 

\section{\label{sec:berry_phase} Berry phase}
Notion of holonomy in the phase of Bloch state transported in the Brillouin zone around a closed loop under the effect of the Berry connection.

\section{\label{sec:kramers} Kramer's theorem}
\begin{enumerate}
    \item Definition of the time reversal operator.
    \item Description of the relation between pairs of opposite Bloch momentum in the Brillouin zone for time reversal symmetric systems.
    \item Introduction of the notion of a Kramer pair of Bloch states and of time reversal invariant momenta. 
    \item Summary of the main consequences of the existence of time reversal moments on the band structure. (Topology of edge modes)\\[0.5cm]
\end{enumerate}

\section{\label{sec:z2_invariant} Invariants and Classification}
\begin{enumerate}
    \item Description of the $\mathbb{Z}_2$ invariant its relation with the Berry phase. 
    \item Summary of the types of Topological insulator.
    \item Bulk-boundary correspondence.\\[0.5cm]
\end{enumerate}


\section{\label{sec:ti_exemples} Topological insulators exemples}
\begin{enumerate}
    \item Description of simple models for two time reversal invariant topological insulators in 2 and 3 dimensions.
    \item Link with real materials.
\end{enumerate}

\subsection{\label{sec:ti_exemples_2D} 2D}
\subsection{\label{sec:ti_exemples_3D} 3D}

\section{\label{sec:concl} Conclusion}
\begin{enumerate}
    \item Opening on other topological systems (topological Superconductivity and charge pumps)\\[0.5cm]
\end{enumerate}
The study of topological insulators is a beautiful application of topology to solid-state physics. It gives profound insight on key properties of some material and some effects using what could've been considered abstract math. Some powerful concept such as the berry phase and Kramer’s theorem rapidly come into play. These concepts allow understanding otherwise very hard to understand yet important phenomena such has Hall effect. The mix of topological and solid-state physics physics howerver doesn't end there. There are other kinds of topological material such as topological \textit{superconductors}. Topological superconductivity, however, has been observed in far fewer materials. Such material still has important applications such as \textit{protected quantum computation}. \cite{nayak_evidence_2021} A lot of research of these materials will likely be done in the upcomming years.
}

\cite{sato_topological_2017}
\cite{ghadimi_topological_2021}
\cite{xiao_berry_2010}
\cite{hasan_topological_2010}
\cite{noauthor_topological_2013}
\cite{sun_quantum_2017}
\cite{chang_x_nodate}
\cite{kane_topological_2013}
\cite{bisharat_photonic_nodate}
\cite{ando_topological_2013}

\bibliographystyle{unsrt}
\bibliography{main.bib}


\end{document}
