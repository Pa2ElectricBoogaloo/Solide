\input{sections/params}

\begin{document}

\preprint{APS/123-QED}

\title{Topological Insulators} 

\author{Pierre-Antoine Graham}
\author{Jean-Baptiste Bertrand}

\altaffiliation{Physics Department, Sherbrooke University.}

\date{\today}

\begin{abstract}
The feild of topological insulators is a recent and very broad. This review aims to be an introduction to the topic. It covers the basic topological concept required to understand the theory such as the idea of topological equivalence, kramer's theorem and the berry phase. Some of the earliest and simplest application of the theorie  are quickly explained. These cover the Integer Quatum and Quantum Spin Hall effect, and the exemple of the HGTE/CDTE heterostructure.
\end{abstract}

\maketitle

\tableofcontents

%\listoffigures


\section{\label{sec:intro} Introduction}
One of the earliest breakthrews allowed by quantum mechanics is the descritpion of metals and insulators with band theory \cite{hoddeson_development_1987} which saw light in 1930. This theory brought a microscopic understanding of the disctinction between these two classes of materials and led to incredible tecnological advances such as the discovery of the transistor by John Bardeen and Walter Brattain in 1947 \cite{brinkman_history_1997}. Beyond its increadible success,  band theory rapidly came in contact with a myriad of intriguing quantum mechanical effects such as the integer quantum Hall effect discovered in 1980 \cite{klitzing_new_1980}. In 1982, Thouless et al. \cite{cayssol_topological_2021} figured out the topological nature of the effect and, in term, brought topology closer to band theory. Altough the integer quantum Hall effect (QH) requires a strong external magnetic field, it was theorised in 2005 by Charles Kane and Eugene Mele \cite{kane_quantum_2005} that similar topological properties could be intrinsicaly realised  through the quantum spin Hall effect (QSH) \cite{qi_quantum_2010}. Experiment then showed in 2007 \cite{koenig_quantum_2007} that HgTe/CdTe
quantum wells (mercury telluride heterostructure) could produce a QSH effect. The theory and experiment of QSH effect led to a deeper classification of solids with topological band theory \cite{soluyanov_topological_nodate}. When applied to insulators, the upgraded band theory creates a seperation between the trivial and the \textit{topological insulators} (TI). The latter is generally caracterised by a metallic boundary and an insulating bulk \cite{moore_birth_2010} as opposed to trivial insulators which are insulating everywhere. The present review will focus on basic properties of time reversal symetric topological insulators based on the mercury telluride example. Sec.\ref{sec:kramers} presents an overview of important ideas from topological band theory. In sec.\ref{sec:QSH}, the main properties of the QSH state are given and compared to the QH effect. Finally, a model of HgTe/CdTe
quantum is studied in sec. \ref{sec:ti_exemples}. 

\section{\label{sec:kramers} Element of topological band theory}
This sections aims to describe the notion of topological invariant and its consequence on the band structure. 
\subsection{Topological equivalence of insulators}
In a periodic lattice potential, electrons are described by bloch states $\ket{n, \mathbf{k}}$ where $n$ is a dicrete quantum number and $\mathbf{k}$ is the crystal momentum in the brillouin zone \cite{shankar_topological_2018}. Each of those states is associated to an energy $E_{n, \mathbf{k}}$.  As it varies with $\mathbf{k}$, the energy sweeps a continous range called a band labeled with the number $n$. Bands are often separated by energy gaps where there are no associated states. Trivial insulators have a gapped ground state meaning that low energy excitations are forbidden by the presence of the gap. On the contrary, topological insulators have metallic (gapless edge states)\cite{kane_topological_2013} and different from trivial insulators in a fundamental way.\\ 

Slow modifications of the hamiltonian of a trivial insulator will change its band structure while leaving it in its ground state. The \textit{deformation} is said to yield equivalent insulators if the gap doesn't close \cite{kane_topological_2013}. This equvalence is topological in the same way the continuous disformation from a torus to a coffee mug is. Just like the coffee mug cannot be continuously disformed into a sphere, a trivial insulator cannot be continuously disformet into a topological insulator.

\subsection{Time reversal symmetry}



\subsection{$\mathbb{Z}_2$ invariant}
%\begin{enumerate}
%    \item Definition of the time reversal operator.
%    \item Description of the relation between pairs of opposite Bloch momentum in the Brillouin zone for time reversal symmetric systems.
%    \item Introduction of the notion of a Kramer pair of Bloch states and of time reversal invariant momenta. 
%    \item Summary of the main consequences of the existence of time reversal moments on the band structure. (Topology of edge modes)\\[0.5cm]
%\end{enumerate}

\section{\label{sec:QSH} Hall Effects}
One of the defining properties of topological insulators is the presence of conducting edge states with an insulating bulk. The example considered here in sec. \ref{sec:ti_exemples} is a two-dimensional material and its edge is therefore one-dimensional. The QH and QSH effects both involve one-dimensional conduction. In one dimension, electrons can only either move forward or backward on the edge of the sample and this restriction is central for Hall effects. \cite{qi_quantum_2010}. 
\subsection{QH}
The QH effect occurs in two-dimensional semiconductor electron gas with a strong applied magnetic field \cite{qi_quantum_2010}. In such a system, electrons occupy \textit{Landau levels} which roughly correspond to a cyclotron motion with a certain \textit{radius}.\cite{laughlin_quantized_1981}

\begin{figure}[h]
    \includegraphics[width=\columnwidth]{sections/visuel/Hall_effect.png}
    \caption{Hall Effect measurements. The upper curve represents the Hall resistance ($R_h \propto \frac{1}{\sigma_{xy}}$) and shows its plateaus while the lower one represent $R_{xx} \propto \rho_{xx}$ and shows its vanishing. Both curves are a function of the magnetic field. \cite{jeckelmann_quantum_nodate}}
    \label{fig:Hall_effet}
\end{figure}

This effect consists of a two simultaneous and related phenomena. The quantized \textit{Hall conductivity} $\sigma_{xy}(=\frac{I_x}{V_y})$ and the vanishing of the $\rho_{xx}$ resistivity in a 2D electron gas. The vanishing of $\rho_{xx}$ is due to the existence of edges states.%because the conducting edge states only carry current in one direction on a given side of the sample.
Such edge state exists because of the discontinuity between the Chern number of the material and the Chern number of empty space. The symmetry breaking region that is the edge of the material implies an unusual asymmetric dispersion relation, which has forward-moving states then backward moving state, imposing conduction along the edges. One can show that the difference between the number of forward  propagating state and the number of backward propagating states at the Fermi level must always be the difference between the Chern numbers of the two material\cite{kane_topological_2013}. That is: $$N_f -N_b = \mathcal{C}$$ These states are resilient to impurities since adding them can't lead to backscattering of the electrons because there are simply no available backward moving states they can scatter into \cite{qi_quantum_2010}. This means that these states have perfect conductivity. Not so surprisingly, the $\sigma_{xy}$ plateaus can also be directly linked to the Chern number. The link is fairly direct since the Hall Conductivity is directly proportional to the Chern number! In fact, we have.
\begin{equation}
\sigma_{xy} = \mathcal{C}\frac{e^2}{\hbar}
\end{equation}

where $\mathcal{C} \in \mathbb{N}$ is the Chern number. This connection between the Chern number and Hall conductivity is made more apparent by considering a third quantity: The number of accessible Landau levels to the system $\nu$ . Since the Chern number is proportional to the number of filled bands (they are summed over to find $\mathcal{C}$) and Landau levels act like bands in this context \cite{kane_topological_2013}, it can be inferred that $\mathcal{C}\propto \nu$. Also, if $\Delta V$ is the potential difference generated by the Hall effect (for a constant current) and $\Delta n_e$ is the number of electrons moved from one side of the material to the other in order to create the potential difference then $\rho_{xy} \propto \Delta V \propto \Delta n_e$. It is then possible to show that the number of electron that you can transfer from one side to the other is direcly proportional to the number of Landau level accessible meaning $\sigma_{xy} \propto \cal{C}$. 

% \begin{figure}[h!]
%     \includegraphics[scale = 0.7]{sections/visuel/spinless.png}
%     \caption{Schematic representation of the conduction chanels in a spinless quantum Hall system.\cite{qi_quantum_2010}}
%     \label{spinless}
% \end{figure}


\begin{figure*}[t]
    \includegraphics[width=\textwidth]{sections/visuel/Hg}
    \centering
    \caption{Schematic representation of the energies of the $\Gamma_6$ and $\Gamma_8$ bands in the \ch{HgTe}/\ch{Hg_xCd_{1-x}Te} quantum well structure \cite{bernevig_topological_2013}. The  $E1$ well level is the highest confined energy level from $\Gamma_6$ and the $H1$ level is the lowest confined level from $\Gamma_8$. $d$ is the thickness of the well and $d_c$ is the critical value for which $E1$ and $H1$ are reversed.}
    \label{hg}
\end{figure*}

\subsection{QSH}

The QSH state can be obtained from the combination of two copies of the QH state \cite{buhmann_quantum_2011}. In this new state, two counter-propagating channels of edge states exist on each side of the sample. Kramer's theorem (see sec. \ref{TRS}) relates these pairs of edge states with TRS \cite{buhmann_quantum_2011}. Since these states are related by TRS, they have opposite crystal momentum $\mathbf{k}$ and spin: they are \textit{helical} edge states \cite{bernevig_topological_2013}. Since helical states couple $\mathbf{k}$ with spin, QSH states can only be realized in systems with strong spin-orbit coupling \cite{qi_quantum_2010}. The number of Kramer pairs of edge states is an example of an $\mathbb{Z}_2$ invariant which tells if the state is topological or not \cite{koenig_quantum_2008}. Here, the helical states come in one Kramer pair (taking effect on both sides of the sample), so there is an odd number of Kramer pairs and the $\mathbb{Z}_2$ invariant allows calling the QSH state a topological state of matter.\\

The impossibility of backscattering on impurities in the QSH state can be directly related to the effect of an antireflection lens \cite{qi_quantum_2010}. The effect of scattering on an impurity is represented in fig. \ref{lens} (B). Two quantum paths can be taken by the electron encountering the impurity. In both cases a helical state must be converted into its Kramer partner by the scattering in order to reverse the direction of motion. When the electron goes around the impurity, its spin turns by an angle of $\pm \pi$ depending on the sense of rotation around the impurity (fig. \ref{lens} (B) up and down shows the two possible paths swapping momentum direction). The global angle difference between the two spin rotations is $2\pi$. Because the phase of half-integer particles rotates with half the speed of the actual rotation witnessed by the electron, the $2\pi$ angle difference becomes a $\pi$ phase shift. A completely destructive interference between the two quantum paths emerges \cite{barut_path_1992} and the backscattering is forbidden. This is just like perfect transmission of an antireflection lens generated by the destructive interference of the red and blue paths of fig. \ref{lens} (A). 


\begin{figure}[t]
    \includegraphics[width=\columnwidth]{sections/visuel/lens.png}
    \caption{Analogy between an antireflection lens (A) and the interference phenomena (B) occurring in the QSH state to prevent backscattering on impurities represented by $X$ \cite{qi_quantum_2010}}
    \label{lens}
\end{figure}



% \begin{figure}[h!]
%     \includegraphics[scale = 0.7]{sections/visuel/spinful.png}
%     \caption{Schematic representation of the conduction chanels in a spinlful quantum Hall system.}
%     \label{spinful}
% \end{figure}

% Talk about the fact that the states are topologicaly protected since their edge states survive the addition of impurities \cite{kane_this_2011}

%\section{\label{sec:berry_phase} Berry phase}
%Notion of holonomy in the phase of Bloch state transported in the Brillouin zone around a closed loop under the effect of the Berry connection.



%\section{\label{sec:z2_invariant} Invariants and Classification}
%\begin{enumerate}
%    \item Description of the $\mathbb{Z}_2$ invariant its relation with the Berry phase. 
%    \item Summary of the types of Topological insulator.
%    \item Bulk-boundary correspondence.\\[0.5cm]
%\end{enumerate}


\section{\label{sec:ti_exemples} HgTe/CdTe heterostructure}


As it was mentionned in sec. \ref{sec:QSH}, the QSH effect requiers strong spin-orbit coupling to occur. Kane and Mele initially proposed that the effect could be realised in graphene but the spin-orbit coupling of carbon is too weak \cite{cayssol_topological_2021}. The first observed topological insulator with QSH effect is a mercury telluride heterostructure consisting of a staking of thin \ch{HgTe} layers between \ch{Hg_xCd_{1-x}Te} \cite{kane_this_2011}. The Band structure of \ch{HgTe} and \ch{Hg_xCd_{1-x}Te} both contain a $\Gamma_6$ and $\Gamma_8$ band \cite{bernevig_topological_2013}. While the $\Gamma_6$ band is $s$-type, the $\Gamma_8$ band is $p$-type. A $s$-type (resp. $p$-type) is formed with the hybridization of $s$ (resp. $p$) orbitals with $0$ (resp. $1$) angular momentum quantum number \cite{girvin_modern_2019}. Adding spin leads to $1/2$ and $3/2$ respective total angular momentum quantum number for $\Gamma_6$ and $\Gamma_8$ bands \cite{bernevig_topological_2013}. In \ch{HgTe}, the strong spin-orbit coupling leads the $\Gamma_8$ band to be more energetic than the $\Gamma_6$ band and the opposite is true for \ch{Hg_xCd_{1-x}Te} \cite{buhmann_quantum_2011}. A schematic view of the resulting quantum well structure is presented on fig. \ref{hg}.


The most important feature of this structure for the realisation of the QSH effect is the fact it the levels $E1$ and $H1$ get inverted for sufficient well thickness $d$ (greater than the critical thickness $d_c$). Band inversion is the signature of the appearence of a topologial phase with variation of the thickness \cite{bansil_colloquium_2016}. Near the edges of the sample, translationnal symmetry breaks and the spin-orbit coupling loses intensity so that the $E1$ and $H1$ levels are inverted back. Gapless edge states emerge from this inversion (see fig. \ref{cross}) and they precisely correspond to the helical Kramer pair states mentionned in sec. \ref{sec:QSH} \cite{buhmann_quantum_2011}.  

\begin{figure}[h!]
    \includegraphics[width=\columnwidth]{sections/visuel/cross}
    \centering
    \caption{Reinversion of the $E1$ and $H1$ levels near the edge of the sample of width $w$.}
    \label{cross}
\end{figure}
\section{\label{sec:concl} Conclusion}

The study of topological insulators is a beautiful application of topology to solid state physics. It gives profond insigh on key propreties of some material and some effet using what could've been considered abstract math. Some powerful concept such as the berry phase and kramers theorem rapidly come into play. These concept allow to understand otherwise very hard to understand yet important phenomena such has Hall effet. The mix of topological and solid state physics physics howerver doesn't end there. There are other kind of topological material such as topological \textit{superconductors}. Topological superconductivity, however, has been observed in far fewer materials. Such material still have important aplication such as \textit{protected quantum computation}. \cite{nayak_evidence_2021} A lot of research of these material will likely by done in the upcomming years.

The study of topological insulators is a beautiful application of topology to solid-state physics. It gives profound insight on key properties of some material and some effects using what could've been considered abstract math. Some powerful concept such as the berry phase and Kramer’s theorem rapidly come into play. These concepts allow understanding otherwise very hard to understand yet important phenomena such has Hall effect. The mix of topological and solid-state physics physics howerver doesn't end there. There are other kinds of topological material such as topological \textit{superconductors}. Topological superconductivity, however, has been observed in far fewer materials. Such material still has important applications such as \textit{protected quantum computation}. \cite{nayak_evidence_2021} A lot of research of these materials will likely be done in the upcomming years.


\bibliographystyle{unsrt}
\bibliography{main.bib}


\end{document}
